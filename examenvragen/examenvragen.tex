\documentclass[11pt,a4paper]{article}
\usepackage[utf8]{inputenc}
\usepackage[dutch]{babel}
\usepackage{amsmath}
\usepackage{amsfonts}
\usepackage{amssymb}
\usepackage{graphicx}
\usepackage[a4paper,bindingoffset=0.2in,%
            left=1in,right=1in,top=1in,bottom=1in,%
            footskip=.25in]{geometry}

\author{Ward Schodts}
\title{Examenvragen - Modellering \& Simulatie}
\begin{document}
\section{Numerieke lineaire algebra en toepassingen}
\subsection{Projectie}
\subsubsection*{Bespreek definitie en eigenschappen van een projector.}

\subsubsection*{Wat is een complementaire projector?}

\subsubsection*{Hoe kan men, gegeven een basis van een deelruimte, een orthogonale projector opstellen? Illustreer aan de hand van de deelruimte: $$ \langle \begin{bmatrix}
1 \\ 1 \\ 1
\end{bmatrix}, 
\begin{bmatrix}
0\\ 1\\ -1
\end{bmatrix} \rangle$$}

\subsection{QR-factorisatie}

\end{document}